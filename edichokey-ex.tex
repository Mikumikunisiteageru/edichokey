%% edichokey-ex.tex
%% Copyright 2017--2020 Yuchang Yang < yang.yc.allium@gmail.com >
%
% This work may be distributed and/or modified under the
% conditions of the LaTeX Project Public License, either version 1.3c
% of this license or (at your option) any later version.
% The latest version of this license is in
%   http://www.latex-project.org/lppl.txt
% and version 1.3c or later is part of all distributions of LaTeX
% version 2005/12/01 or later.
%
% This work has the LPPL maintenance status `maintained'.
% 
% The Current Maintainer of this work is Yuchang Yang.
%
% This work consists of:
%   - the style file: [edichokey.sty];
%   - the manual files: [edichokey-doc-en.tex, edichokey-doc-en.pdf, README.md];
%   - the example files: [edichokey-ex.tex, edichokey-ex.pdf].
%%

\documentclass[a4paper]{article}

\usepackage[margin=25mm]{geometry}
\usepackage{edichokey}

\def\var{\textup{var.} }
\frenchspacing
\tolerance2000

\begin{document}

\vspace*{1cm}
\centerline{\Large Identification key to \textit{Allium} species in North America}
\vskip8mm

\begingroup
\leftskip1em\rightskip1em
\noindent Source: \textsc{McNeal Jr., Dale W., \& Jacobsen, T. D.} 2002. \textit{Allium} Linnaeus. \textit{Pages 224--275 of:} \textsc{Flora of North America Editorial Committee} (ed), \textit{Flora of North America North of Mexico}, vol. 26. New York and Oxford: Oxford University Press.\par\endgroup

\vskip8mm

\begin{Key}{A. }
\alter Leaf blade flat, channeled, or ± terete, never more than 30 mm wide, (never petiolate).
\alter Flowering pedicels mostly or completely replaced by bulbils.
\alter Outer bulb coats persisting as fibrous reticulum; leaf sheaths not extending more than 1/4 scape; spathe bract beakless or beak much shorter than base.
\alter Ovary, when present, crestless; spathe bracts 3--7-veined; east of 103rd meridian
\name{canadense}
\alter Ovary, when present, obscurely crested with 6, low, central processes; spathe bracts 1-veined; west of 105th meridian
\name{geyeri}
\alter Outer bulb coats membranous, if with fibers these not forming reticulum; leaf sheaths extending to midscape or above; spathe bract with beak equaling or longer than base.
\alter Spathe bract 1, caducous.
\alter Bulbs 1--2 cm diam.; leaf blade 2--4 mm diam., cylindric or filiform, not carinate, hollow below middle
\name{vineale}
\alter Bulbs (1.5--)3--8 cm diam.; leaf blade 5--20 mm wide, flat, carinate, solid
\name{sativum \var sativum}
\alter Spathe bracts 2--5, persistent.
\alter Spathe bracts 2--5, 4--9-veined, beak to 20 cm
\name{oleraceum}
\alter Spathe bracts 3--5, 2--3-veined, beak to 10 cm
\name{ampeloprasum}
\alter Flowering pedicels floriferous, bulbils almost unknown.
\alter Outer bulb coats persisting as fibrous reticulum.
\alter Ovary usually crestless; if obscurely crested, with 3 or 6 processes; east of 103rd meridian.
\alter Spathe bracts usually 1-veined.
\alter Spaces between bulb coat fibers filled in proximal 1/2 bulb; tepals white, pink, or red, rarely greenish yellow; central plains from N Mexico to Nebraska
\name{drummondii}
\alter Spaces between bulb coat fibers open; tepals yellow; W Texas
\name{coryi}
\alter Spathe bracts 3--7-veined.
\alter Umbel compact; pedicels much shorter than flowers
\name{schoenoprasum}
\alter Umbel loose; pedicels longer than flowers.
\alter Flowers substellate to urceolate-campanulate, ultimately withering somewhat and exposing capsule; reticula of bulbs finely or only moderately coarsely meshed.
\alter Bulbs 1--3, narrowly cylindric, attached to ± horizontal primary rhizome, often missing or not visible on herbarium specimens; leaf blade carinate; cells of seed coat smooth, shiny; occasional introduction
\name{tuberosum}
\alter Bulbs 1--4+, ovoid, not attached to rhizome; leaf blades not carinate, channeled; cells of seed coat each with minute, central papilla; native east of 103rd meridian
\name{canadense}
\alter Flowers urceolate, permanently investing capsule; reticula of bulbs usually very coarsely meshed.
\alter Flowering bulbs with cluster of stalked, basal bulbels; cells of innermost bulb coats contorted, with sinuous walls; extreme S Texas
\name{runyonii}
\alter Flowering bulbs without basal bulbels; cells of innermost bulb coats vertically elongate, without sinuous walls; W Texas and E New Mexico to C South Dakota
\name{perdulce}
\alter Ovary usually crested with 3 or 6 processes; if crestless, from west of 105th meridian.
\alter Ovary and capsule conspicuously crested with 6 contorted or horizontally spreading, ± lateral processes; tepals widely spreading to reflexed, SE United States.
\alter Spathe bracts usually 5--7-veined; ovary crests conspicuously contorted; tepals spreading to reflexed
\name{cuthbertii}
\alter Spathe bracts 1-veined; ovary crests flattened, horizontally spreading, not contorted; tepals widely spreading
\name{speculae}
\alter Ovary crested with 6 ± erect, often obscure central processes; tepals erect to widely spreading; W North America.
\alter Leaves 3+ per scape; cells of seed coat each with minute, central papilla.
\alter Bulbs often short-rhizomatous basally; spathe bracts 3--5-veined; ovary conspicuously crested with 6 flattened, lacerate central processes; tepals spreading or reflexed, withering in fruit, not investing capsule
\name{plummerae}
\alter Bulbs not short-rhizomatous; spathe bracts usually 1-veined; ovary obscurely crested with 6 rounded central processes; tepals erect, not withering in fruit, permanently investing capsule.
\alter Leaf blade flat, ± falcate, usually 3--6 mm wide; Box Elder County, Utah
\name{passeyi}
\alter Leaf blade channeled, ± straight, usually less than 5 mm wide; widespread, N Great Plains and W North America
\name{geyeri}
\alter Leaves usually 2 per scape; cells of seed coat ± smooth, with or without central papillae.
\alter Spathe bracts 3--5-veined; tepals becoming papery in fruit, midrib scarcely thickened, not investing capsule; ovary usually conspicuously crested with 6 flattened central processes, often to 2 mm
\name{macropetalum}
\alter Spathe bracts 1-veined; tepals becoming callous-keeled, permanently investing capsule; ovary inconspicuously crested with 6 rounded central processes, to 1 mm.
\alter Leaf blade flat, ± falcate, usually 3--6 mm wide; cells of seed coat with minute central papilla; Box Elder County, Utah
\name{passeyi}
\alter Leaf blade semiterete, channeled, ± straight, usually 1--3(--5) mm wide; cells of seed coat smooth; N Great Plains and W North America
\name{textile}
\alter Outer bulb coats membranous to chartaceous, with or without distinct cellular markings (reticulation); without fibers or with some parallel fibers.
\alter Scape fistulose, 3--25 mm diam., not flattened and winged; leaves 2--10, blade flat and solid, or fistulose.
\alter Leaf blade flat, solid.
\alter Leaves not or scarcely sheathing base of scape
\name{nigrum}
\alter Leaves sheathing 1/3--1/2 scape
\name{ampeloprasum}
\alter Leaf blade fistulose.
\alter Bulbs 1--3, to 10 cm diam., ± globose, not rhizomatous; leaf blade semicircular in cross section; occasional escape from cultivation
\name{cepa}
\alter Bulbs 1--2, 5 cm diam., cylindric, clustered on short rhizome (this often missing or not visible on herbarium specimens); leaf blade circular in cross section; native or introduced.
\alter Flowers 8--18 mm; tepals lilac to pale purple; native or introduced
\name{schoenoprasum}
\alter Flowers 6--9 mm; tepals pale yellowish white; introduced
\name{fistulosum}
\alter Scape solid, exceeding 5 mm wide only if flattened and winged; leaves 1--several, leaf blade solid.
\alter Leaves (3--)5--40 mm wide, basal sheaths extending 1/3--1/2 scape.
\alter Filaments unappendaged; leaf blade terete to semiterete; bulbels, if present, light brown
\name{paniculatum}
\alter Inner filaments appendaged with prominent tooth on each side of anther; leaf blade flat, channeled; bulbels very dark purple
\name{rotundum}
\alter Leaves 1--25 mm wide, basal sheaths never extending much above soil level.
\alter Bulbs oblong, elongate, or ovoid, clustered on stout, primary rhizome, or short-rhizomatous; bulb coats membranous or chartaceous, finely striate with narrow, vertically elongate cells.
\alter Bulbs on stout, iris-like rhizome; ovary crestless.
\alter Tepals elliptic, apex obtuse; stamens ± equaling tepals; EC Arizona and adjacent New Mexico, and Santa Catalina Mountains, S Arizona
\name{gooddingii}
\alter Tepals narrowly lanceolate to lanceolate, apex acuminate; stamens much shorter than tepals or definitely exserted; widespread in W North America, not occurring in Arizona.
\alter Stamens and style exserted; stigma capitate; Cascades and Sierras E to NE Nevada, E Oregon, W Idaho
\name{validum}
\alter Stamens and style ca. 1/2 tepals; stigma 3-lobed; Rocky Mountains from C Montana and NE Idaho to Wyoming, NE Utah, Colorado, and New Mexico
\name{brevistylum}
\alter Bulbs short-rhizomatous at base, rhizome not stout and iris-like; ovary strongly crested with 6 processes.
\alter Stamens and styles included; outer bulb coats ± reddish brown, inner coats deep red to white; ovary crested with 6 short, rounded, densely papillose processes
\name{haematochiton}
\alter Stamens and styles exserted; outer bulb coats gray or brown, inner coats white to pink or reddish; ovary crested with 6 flattened, ± triangular processes, margins entire or toothed.
\alter Flowers campanulate; tepals ± erect; scape nodding
\name{cernuum}
\alter Flowers stellate; tepals spreading; scape erect, or, if nodding at anthesis, becoming erect
\name{stellatum}
\alter Bulbs ovoid to subglobose, not clustered on stout, primary rhizome; rhizomes, if present, secondary, arising from bulbs, ± slender, terminated by new bulbs; bulb coats without reticulation or with ± isodiametric or transversely elongate cells that are sometimes intricately contorted.
\alter Leaf 1 per scape; leaf blade terete; ovary prominently crested with 6 ± triangular processes.
\alter Stigma unlobed or minutely 3-lobed, lobes ± stout, erect or spreading.
\alter Scape 18--60 cm; flowers 5--9 mm; tepals unequal, inner whorl 1/4--1/3 longer than outer, margins entire or irregular to erose; stamens exserted
\name{sanbornii}
\alter Scape less than 25 cm; flowers 7--20 mm; tepals ± equal, margins entire; stamens included.
\alter Outer bulb coat reticulate with ± elongate, contorted meshes
\name{nevadense}
\alter Outer bulb coat lacking reticulation, or meshes very indistinct, square or polygonal.
\alter Pedicels slender, longer than flowers; flowers 8--12 mm
\name{atrorubens}
\alter Pedicels stout, generally shorter than flowers; flowers 12--20 mm.
\alter Tepals lanceolate to lance-linear, apex acute; lacking stalked, basal increase bulbs; rocky, sandy desert slopes, S California to W Arizona
\name{parishii}
\alter Tepals lance-linear to lanceolate, apex long-acuminate; with 1--2 stalked basal increase bulbs; alpine ridges and talus, S California mountains
\name{monticola}
\alter Stigma distinctly 3-lobed, lobes often slender and recurved.
\alter Stamens equaling tepals or exserted.
\alter Tepals unequal, inner 1/3--1/2 longer than outer
\name{sanbornii}
\alter Tepals ± equal
\name{howellii}
\alter Stamens included.
\alter Tepal (at least inner whorl) margins denticulate to erose.
\alter Scape 25--40 cm
\name{jepsonii}
\alter Scape 5--20 cm.
\alter Outer bulb coats reddish brown; tepals erect, ± straight at tip; inner whorl margins denticulate
\name{denticulatum}
\alter Outer bulb coats brown to gray; tepals erect, ± spreading-reflexed at tip; inner whorl margins denticulate to erose
\name{abramsii}
\alter Tepal margins all ± entire.
\alter Margins of ovarian crest processes entire or notched at tip, outer margins sometimes irregular but never dentate or laciniate.
\alter Flowers 10--18 mm; tepals maroon or deep reddish purple.
\alter Tepals deep reddish purple, all reflexed at tip; Mount Hamilton Range, C California
\name{sharsmithiae}
\alter Tepals maroon, outer curled back at tip, inner reflexed; Spanish Needle Peak, S Sierra Nevada, and Horse Canyon, Tehachapi Mountains, California
\name{shevockii}
\alter Flowers 6--9 mm; tepals white to pink, darkening in age.
\alter Inflorescence loose; pedicels flexuous in fruit; tepals lanceolate to lance-ovate, apex acuminate
\name{parryi}
\alter Inflorescence compact; pedicels straight; tepals ovate to nearly round, apex obtuse (rarely acute) to shallowly emarginate
\name{munzii}
\alter Margins of ovarian crest processes dentate to laciniate.
\alter Tepals deep reddish purple, erect, usually conspicuously recurved at tip
\name{fimbriatum}
\alter Tepals white or flushed to pale lavender with darker midveins, spreading or erect, not conspicuously recurved at tip.
\alter Flowers usually 6--12 mm
\name{fimbriatum}
\alter Flowers usually 6--8(--10) mm.
\alter Scape 25--50 cm; tepals spreading from base; serpentine soil, Rawhide Hill and Red Hills, foothills of Sierra Nevada, C California
\name{tuolumnense}
\alter Scape 7--20(--30) cm; tepals erect; serpentine clay soils, S Coast Ranges and W Transverse Ranges, California
\name{diabolense}
\alter Leaves usually 2 or more, if 1, blade flattened or broadly channeled; ovary crestless or variously crested.
\alter Bulbs generally with numerous increase bulbs, these much smaller than parent bulb, enclosed by bulb coats, in basal cluster or on threadlike rhizomes to 10 cm.
\alter Ovary crestless or obscurely crested with 3 low central processes.
\alter Larger bulbs each with cluster of bulbels surrounding roots; S Texas
\name{elmendorfii}
\alter Larger bulbs each with cluster of small, basal bulbels on one side; NE Oregon and WC Idaho
\name{madidum}
\alter Ovary prominently crested with 6 triangular central processes, margins finely papillose or denticulate.
\alter Leaves usually beginning to wither from tip by anthesis; tepals rigid (not papery), ± shiny in fruit, strongly involute at tip, carinate
\name{campanulatum}
\alter Leaves usually green at anthesis; tepals papery (not rigid and shiny) in fruit, not strongly involute, not carinate.
\alter Tepals ovate to elliptic, apex acute to acuminate; foot\-hills of Sierra Nevada, N, C California
\name{membranaceum}
\alter Tepals lanceolate, apex acuminate; Sierra Nevada, California, and intermountain region N to Oregon, Idaho
\name{bisceptrum}
\alter Increase bulbs absent or 1--4, ± equaling parent bulbs, enclosed by parental bulb coats, never appearing as basal cluster, not rhizomatous or rhizomes 2+ mm thick (not threadlike).
\alter Leaf blade channeled to subterete, if flat, not falcate.
\alter Bulb coats lacking reticulation or reticulum delicate, very obscure under hand lens.
\alter Bulbs ovoid to subglobose; rhizomes absent, renewal bulbs formed within coats of parent bulb; native or introduced.
\alter Scape terete throughout, 1--3 mm diam.; leaf blade 1--3 mm wide; native to W Texas to SE Arizona
\name{kunthii}
\alter Scape triquetrous, 2-edged or slightly winged proximally, if terete only proximally so, 1--10 mm wide; introduced in California and Oregon near the Pacific coast.
\alter Umbel erect, ± hemispheric; flowers ± erect; tepals broadly elliptic, apex obtuse
\name{neapolitanum}
\alter Umbel lax, ± 1-sided; flowers pendent; tepals lanceolate, apex acute
\name{triquetrum}
\alter Bulbs oblique or oblique-ovoid, renewal bulbs borne terminally on rhizomes outside coats of parent bulbs; native.
\alter Rhizomes conspicuous, 2 cm or more, including renewal bulbs.
\alter Rhizomes smooth, parent bulb disappearing by anthesis except for still-functional roots and bulb coat; leaf blade broadly concave-convex or ± flattened, carinate; tepals obovate to ovate, apex acute to obtuse or emarginate; Coast Ranges, California, Oregon
\name{unifolium}
\alter Rhizomes scaly, sometimes absent, often missing in herbarium specimens, parent bulb persisting after anthesis; leaf blade flat, not carinate; tepals lanceolate to oblong, apex acute to acuminate; trans- Pecos Texas to SE Arizona
\name{rhizomatum}
\alter Rhizomes inconspicuous, 2 cm or less, including renewal bulb.
\alter Tepals erect, red-purple, rarely pure white, at least inner tepal margins serrulate; NW California, SW Oregon
\name{bolanderi}
\alter Tepals ± spreading, white to pale pink, margins entire; W Texas to SE Arizona
\name{kunthii}
\alter Bulb coats obviously reticulate with prominent meshes under hand lens.
\alter Cells of outer bulb coat square or polygonal.
\alter Ovary with 6 prominent, flat, ± triangular crest processes
\name{bigelovii}
\alter Ovary with 3 or 6 minute, rounded crest processes, or crest obscure.
\alter Flowers 4--9 mm; tepals erect or spreading from base, margins entire
\name{lacunosum}
\alter Flowers 8--16 mm; tepals spreading at tip, inner tepal margins denticulate.
\alter Bulb forming 1--3 renewal bulbs borne terminally on rhizomes outside coats of parent bulb; parent bulb disappearing by anthesis except for still-functional roots and shriveled bulb coats; near Weller Butte, Blue Mountains, SE Washington
\name{dictuon}
\alter Bulbs not forming rhizomes, renewal bulbs formed within coats of parent bulb; widespread W of Rocky Mountains
\name{acuminatum}
\alter Cells of bulb coat transversely elongate, V-shaped, arranged in ± vertical rows, forming herringbone pattern, or ± contorted.
\alter Cells of bulb coat in wavy, transverse rows, forming indistinct herringbone pattern or ± contorted; tepals spreading, ± equal.
\alter Scape (3--)5--15(--17) cm; umbel persistent; tepals erect, not connivent over capsule in fruit
\name{hickmanii}
\alter Scape 15--60 cm; umbel shattering, each flower with its pedicel falling as unit; tepals connivent over capsule in fruit.
\alter Ovary crested with 6 ± rectangular lateral processes; umbel compact; pedicel 0.7--2 times perianth
\name{amplectens}
\alter Ovary crestless or crested with 3 minute, 2-lobed central processes; umbel loose; pedicel 1.5--4 times perianth.
\alter Leaf blade to 10 mm wide, channeled or flattened, carinate; inner bulb coats white; tepals becoming papery (not hyaline) after anthesis
\name{praecox}
\alter Leaf blade 1--3 mm wide, channeled or subterete, not carinate; inner bulb coats light yellow or white; tepals becoming hyaline (not papery) after anthesis
\name{hyalinum}
\alter Cells of bulb coat in sharply serrate, transverse rows, forming distinct herringbone pattern; tepals erect, inner shorter, narrower.
\alter Tepals connivent over capsule in fruit, not rigid; umbel shattering in fruit, each flower with its pedicel falling as a unit
\name{serra}
\alter Tepals not connivent over capsule, rigid in fruit; umbel persistent.
\alter Leaves 3--6, blade arcuate to tortuous; umbel compact; pedicels 5--20 mm; sea cliffs, N, C California
\name{dichlamydeum}
\alter Leaves 2--3, blade straight to arcuate; umbel loose; pedicels 10--40 mm; not on sea cliffs, California Floristic Province, extending south in coastal ranges.
\alter Inner tepal margins denticulate, crisped
\name{crispum}
\alter Inner tepal margins entire to denticulate, never crisped
\name{peninsulare}
\alter Leaf blade flat or broadly channeled, if flat, ± falcate.
\alter Scape and leaves persisting after seeds mature or on pressing, or only tardily deciduous.
\alter Stamens much shorter than tepals.
\alter Bulb coat cellular-reticulate with elongate, ± obscure, intricately contorted cells (resembling \textit{Allium madidum}, but never with cluster of basal bulbels)
\name{fibrillum}
\alter Bulb coat cellular-reticulate with ± narrowly hexagonal, transversely elongate cells
\name{brandegeei}
\alter Stamens equaling tepals or exserted.
\alter Scape expanded proximal to inflorescence; leaf blade (2--)5--8 mm wide
\name{columbianum}
\alter Scape thickest immediately proximal to inflorescence; leaf blade 1--5(--15) mm wide.
\alter Scape constricted just proximal to inflorescence, then expanded; leaf blade 1--3(--5) mm wide
\name{constrictum}
\alter Scape not expanded proximal to inflorescence; leaf blade 2--5(--15) mm wide.
\alter Leaf blade usually more than 5 mm wide, flat; umbel 25--50-flowered; spathe bracts 3
\name{douglasii}
\alter Leaf blade 2--3 mm wide, flat to channeled; umbel 10--30-flowered; spathe bracts 2.
\alter Bulb coat with quadrate to polygonal reticulations; leaf blade ± equaling scape
\name{nevii}
\alter Bulb coat without reticulations or with 2--3 rows of ± quadrate cells just distal to roots; leaf blade exceeding scape
\name{macrum}
\alter Scape and leaves forming abcission layer at soil surface and deciduous when seeds mature, also frequently breaking at soil surface after pressing.
\alter Outer bulb coats cellular-reticulate throughout (often obscurely so in \textit{A. aaseae} and \textit{A. simillimum}).
\alter Bulb coats obscurely cellular-reticulate with ± contorted cells; tepal margins denticulate to erose.
\alter Tepals white with greenish or reddish veins, sometimes flushed pink; anthers purple or mottled purple and white; pollen white or gray
\name{simillimum}
\alter Tepals bright pink, rarely white; anthers yellow; pollen yellow
\name{aaseae}
\alter Bulb coats ± prominently cellular-reticulate; tepal margins entire.
\alter Bulb coats reticulate, cells irregularly arranged, ± polygonal, rectangular, or transversely elongate, ± curved.
\alter Cells of bulb coat irregularly arranged, ± transversely elongate, curved; Tuolumne County, C California
\name{tribracteatum}
\alter Cells of bulb coat irregularly arranged or in ± regular vertical rows, polygonal or ± rectangular; Sierra Nevada, California, and Nevada
\name{obtusum}
\alter Bulb coats reticulate, cells arranged in ± regular vertical rows, narrowly hexagonal to rectangular, transversely elongate.
\alter Tepals linear-lanceolate
\name{anceps}
\alter Tepals oblanceolate to ovate.
\alter Scape 3--10 cm; pedicel ± equaling perianth
\name{punctum}
\alter Scape 15--20 cm; pedicel 2--3 times perianth
\name{lemmonii}
\alter Outer bulb coats not cellular-reticulate or with 2--3 rows of cells just distal to roots.
\alter Scape terete or ± compressed, not winged.
\alter Stamens well included.
\alter Stamens ± equaling tepals or exserted.
\alter Leaf blade strongly falcate; umbel mostly 5--10-flowered
\name{parvum}
\alter Leaf blade linear or weakly falcate; umbel 20--30-flowered
\name{cratericola}
\alter Leaves 2 per scape.
\alter Leaf 1 per scape.
\alter Leaf blade ± equaling to 2 times scape; WC Idaho
\name{tolmiei}
\alter Leaf blade much longer than scape; C Sierra Nevada, California
\name{yosemitense}
\alter Filaments papillose proximally
\name{hoffmanii}
\alter Filaments smooth proximally
\name{burlewii}
\alter Scape flattened, 2-edged or usually winged distally.
\alter Bulbs oblique or oblique-ovoid, renewal bulbs borne terminally on rhizomes outside coats of parent bulb; parent bulb disappearing by anthesis except for still-functional roots and shriveled bulb coat.
\alter Pedicel ± equaling perianth; ovary obscurely 3-crested; barren, bald summits W of Cascade Mountains from Vancouver Island to SW Oregon, also at Jefferson Park, Oregon, and in Wenatchee Mountains, C Washington
\name{crenulatum}
\alter Pedicel 2--3 times perianth; ovary prominently 6-crested; mountains and scablands E of Cascade Mountains, Oregon
\name{tolmiei}
\alter Bulbs ovoid to subglobose, rhizomes absent, renewal bulbs formed within coats of parent bulb; parent bulbs persistent.
\alter Tepals narrowly lanceolate, apex long-acuminate; stamens exserted
\name{platycaule}
\alter Tepals lanceolate to ovate or elliptic, apex obtuse to acuminate; stamens included.
\alter Flowers 9--15 mm; tepal apex long-acuminate, inner margins usually denticulate
\name{falcifolium}
\alter Flowers 6--10(--12) mm; tepal apex obtuse to acute, or ± involute in age and appearing acuminate, inner margins denticulate or not.
\alter Inner bulb coats usually pink or red; inner tepal margins sometimes ± denticulate; Siskiyou Mountains of NW California and SW Oregon
\name{siskiyouense}
\alter Inner bulb coats white; inner tepal margins entire; W United States, E of Sierra--Cascade axis.
\alter Tepals becoming rigid (not papery), carinate in fruit.
\alter Tepals lanceolate, apex acute to acuminate, ± erect in fruit, involute at tip; ovary obscurely to prominently crested with 3 or 6 processes
\name{tolmiei}
\alter Tepals elliptic-oblong, apex obtuse, not involute at tip, connivent over ovary in fruit; ovary crestless or obscurely crested
\name{scilloides}
\alter Tepals becoming papery (not rigid), not carinate in fruit.
\alter Ovary distinctly crested with 3 or 6 low processes; sand and gravel deposits, along Columbia River from Ferry County, NE Washington, to mouth of John Day River, NC Oregon
\name{robinsonii}
\alter Ovary obscurely crested with 3 low, rounded processes; rocky, clay slopes and talus, E Oregon, Idaho, to C California, N Nevada, NW Utah
\name{parvum}
\alter Leaf blade flat, 15--90 mm wide, (tapering to base or distinctly petiolate).
\alter Leaves ephemeral, usually absent at anthesis; E North America
\name{tricoccum}
\alter Leaves present at anthesis; Attu and Unalaska islands, Alaska
\name{victorialis}
\end{Key}

\end{document}
