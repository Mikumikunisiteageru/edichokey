% qdichokey/example.tex
% 20171213

\documentclass{ctexart}

\usepackage{qdichokey}
\usepackage[margin=15mm]{geometry}

\begin{document}

\section*{A small key example}

\begin{Key*}{A.~} % Acronym for an anonymous genus
\alter Leaves large \name{macrophyllum}
\alter Leaves small.
	\alter Flowers large.
		\alter Capsules large \name{macrospermum}
		\alter Capsules small \name{microspermum}
	\alter Flowers small \name{microflorum}
\end{Key*}

\section*{Key for \textit{Allium} species in \emph{Flora of China}}

\begin{Key*}{A.~}
\alter Leaves 1--3, linear to orbicular-ovate, base usually narrowed into a petiole; ovary base often constricted into a short stipe; ovules 1 per locule.
\alter Leaf 1, long petiolate, ovate to broadly elliptic-ovate, base cordate
\cname{玉簪叶山葱}{funckiifolium}
\alter Leaves 2 or 3.
\alter Outer perianth segments narrower than inner ones.
\alter Base of leaf blade cuneate, decurrent
\cname{茖葱}{victorialis}
\alter Base of leaf blade rounded to cordate, not decurrent
\cname{对叶山葱}{listera}
\alter Outer perianth segments as wide as or wider than inner ones.
\alter Scape shorter than leaves, 2--5 cm, covered with leaf sheaths for 3/4--4/5 its length
\cname{短葶山葱}{nanodes}
\alter Scape longer than leaves, 10--60 cm, covered with leaf sheaths only at base.
\alter Leaves lanceolate-oblong to ovate-oblong, base rounded to cordate, petiole distinct
\cname{卵叶山葱}{ovalifolium}
\alter Leaves linear, linear-lanceolate, elliptic-lanceolate, elliptic-oblanceolate, or rarely narrowly elliptic, base narrowed, petiole indistinct
\cname{太白山葱}{prattii}
\alter Leaves several, lorate or linear, cross section semiterete or terete, solid or fistulose, base usually not narrowed into a petiole; ovules 2 to several per locule; if leaf base narrowed into a petiole or ovules 1 per locule then bulb tunic never distinctly reticulate.
\alter Roots thick and fleshy, sometimes subtuberous; leaves with distinct midvein; scape usually 2- or 3-angled; ovules 1 or 2 per locule.
\alter Ovules 1 per locule (in \textit{A. omeiense} a few ovaries in same umbel with 2 ovules per locule).
\alter Umbel laxly fascicled, few flowered; pedicels unequal; style much shorter than ovary; stigma 3-cleft
\cname{三柱韭}{trifurcatum}
\alter Umbel hemispheric to globose, many flowered; pedicels equal; style equaling or longer than ovary; stigma entire, punctiform.
\alter Scape terminal
\cname{灌县韭}{guanxianense}
\alter Scape lateral.
\alter Leaves lanceolate to linear-lanceolate, distinctly contracted at base; filaments longer than perianth segments
\cname{乡城韭}{xiangchengense}
\alter Leaves linear or lorate to lorate-oblanceolate, not contracted at base; filaments shorter than to subequaling perianth segments.
\alter Perianth segments 4--7.5 mm, free; filaments slightly shorter than to subequaling perianth segments
\cname{宽叶韭}{hookeri}
\alter Perianth segments 9--11 mm, connate at base into a tube ca. 1 mm; filaments ca. 1/2 as long as perianth segments
\cname{峨眉韭}{omeiense}
\alter Ovules 2 per locule.
\alter Perianth yellow, segments united for ca. 1 mm at base
\cname{剑川韭}{chienchuanense}
\alter Perianth white, red, purple-red, or dark purple, segments free.
\alter Filaments connate into a tube for 2/3--3/4 their length
\cname{杯花韭}{cyathophorum}
\alter Filaments connate only at base.
\alter Perianth white, segments lanceolate, apex acuminate or irregularly 2-lobed
\cname{粗根韭}{fasciculatum}
\alter Perianth red, purple-red, or dark purple, rarely whitish, segments oblong, narrowly so, or ovate-oblong, apex retuse, truncate, or obtuse.
\alter Perianth stellately spreading, reflexed after anthesis, inner and outer segments similar; pedicels straight
\cname{多星韭}{wallichii}
\alter Perianth campanulate, not reflexed after anthesis, inner segments somewhat longer and narrower than outer ones; pedicels nodding at apex
\cname{大花韭}{macranthum}
% \end{Key}\begin{Key}{A.~}
\alter Roots thin; leaves without distinct midvein; scape terete or several angled; ovules 2 to several per locule.
\alter Bulb usually solitary, globose, ovoid-globose, or ovoid (if cylindric to ovoid-cylindric, then leaves thick, terete, and fistulose); rhizomes obscure.
\alter Leaves usually thick, terete, fistulose, smooth.
\alter Bulb flattened globose, globose, or ovoid-globose, rarely cylindric with thickened base; base of inner filaments 1-toothed on each side (if entire, then scape often undeveloped).
\alter Scape solid
\cname{实葶葱}{galanthum}
\alter Scape fistulose.
\alter Scape developed; plants propagated by seeds or bulblets
\cname{洋葱}{cepa}
\alter Scape usually undeveloped; plants propagated by bulbs.
\alter Bulb ovoid-globose to ovoid
\cname{香葱}{cepiforme}
\alter Bulb narrowly ovoid or cylindric-ovoid
\cname{洋葱}{cepa}
\alter Bulb cylindric to ovoid-cylindric; filaments entire.
\alter Filaments shorter than perianth segments, connate into a tube for 1/3--3/4 their length
\cname{蓝苞葱}{atrosanguineum}
\alter Filaments shorter or longer than perianth segments, connate only at base.
\alter Perianth pale red, pale purple, or purple-red.
\alter Pedicels unequal, shorter than to nearly as long as perianth; filaments 1/3--1/2(--2/3) as long as perianth segments
\cname{北葱}{schoenoprasum}
\alter Pedicels subequal, 1.5--3 × as long as perianth; filaments slightly shorter than to longer than perianth segments.
\alter Perianth rose pink or dark pink; filaments slightly shorter than to equaling perianth segments
\cname{马葱}{maximowiczii}
\alter Perianth pale purple; filaments longer than perianth segments
\cname{硬皮葱}{ledebourianum}
\alter Perianth yellow to white.
\alter Leaves and scape ± thin, to 5 mm thick; perianth yellow or pale yellow
\cname{野葱}{chrysanthum}
\alter Leaves and scape robust, more than 5 mm thick; perianth white or yellowish white.
\alter Bulb ovoid-cylindric, robust, tunic red-brown, thinly leathery; perianth yellowish white; pedicels slightly shorter than to 2 × as long as perianth
\cname{阿尔泰葱}{altaicum}
\alter Bulb cylindric, tunic usually white, rarely light red-brown, membranous; perianth white; pedicels 2--3 × as long as perianth
\cname{葱}{fistulosum}
\alter Leaves slender, flat, triangular-flat, semiterete, or rarely terete, fistulose.
\alter Ovules 4 or more per locule.
\alter Perianth segments united into a tube proximally.
\alter Scape (15--)20--50 cm; pedicels (4.5--)7--11 cm; perianth segments 7--10 mm; ovules (5 or)6(--8) per locule
\cname{长梗合被韭}{neriniflorum}
\alter Scape 15--30(--40) cm; pedicels 0.8--4(--7) cm; perianth segments 5--7(--8) mm; ovules (3 or)4 per locule, rarely 1 or 2 locules with 5 or 6 ovules.
\alter Perianth red to purple
\cname{合被韭}{tubiflorum}
\alter Perianth white
\cname{齿棱茎合被韭}{inutile}
\alter Perianth segments free.
\alter Filaments connate and adnate to perianth segments for 1/2--2/3 their length.
\alter Leaves linear, 4--6(--8) mm wide; perianth cupular, segments broadly elliptic, 8--11 × 4--4.5 mm
\cname{高地蒜}{oreophilum}
\alter Leaves broadly linear, (5--)10--25 mm wide; perianth narrowly campanulate, segments linear-oblong, 7--10(--15) × 2.5--3 mm
\cname{伊犁蒜}{winklerianum}
\alter Filaments connate and adnate to perianth segments only at base.
\alter Leaves 1(or 2); perianth segments 6--6.5 mm, without strong midvein; ovary stipitate
\cname{多籽蒜}{fetisowii}
\alter Leaves (1 or)2 or 3; perianth segments 4.5--5.5 mm, with strong midvein; ovary sessile.
\alter Perianth red to purple-red
\cname{健蒜}{robustum}
\alter Perianth white or whitish lilac to lilac-pink.
\alter Leaves 1--1.5(--2) cm wide, apex gradually attenuate
\cname{郁金叶蒜}{tulipifolium}
\alter Leaves 0.7--1.5 cm wide, apex acute
\cname{新疆蒜}{roborowskianum}
\alter Ovules 2 per locule.
\alter Inner filaments 1-toothed on each side at base, apex of tooth long filiform and longer than anther-bearing cusp of filament.
\alter Umbel with flowers only; filaments longer than perianth segments
\cname{韭葱}{porrum}
\alter Umbel with both flowers and bulblets; filaments shorter than perianth segments
\cname{蒜}{sativum}
\alter Inner filaments entire or, if 1-toothed on each side, apex of tooth not filiform and never longer than anther-bearing cusp of filament.
\alter Perianth blue, usually becoming bluish purple when dried.
\alter Leaves flat, abaxially 1-angled, usually twisted when dried
\cname{棱叶薤}{caeruleum}
\alter Leaves semiterete, not twisted when dried
\cname{知母薤}{caesium}
\alter Perianth white, pale red, red, purple-red, purple, or pale green.
\alter Filaments not more than 2/3 as long as perianth segments.
\alter Pedicels ca. 3 × as long as perianth
\cname{高原薤}{jacquemontii}
\alter Pedicels shorter than or subequaling perianth.
\alter Scape 10--15(--20) cm; perianth segments obtuse or attenuate at apex; broadened part of inner filaments with 1 small tooth on each side
\cname{类北薤}{schoenoprasoides}
\alter Scape (15--)25--40 cm; perianth segments acute at apex; broadened part of inner filaments entire
\cname{赛里木薤}{sairamense}
\alter Filaments more than 2/3 as long as perianth segments. 
\alter Plants dioecious; female flowers 1 per scape; male flowers 2--4(or 5) per scape
\cname{单花薤}{monanthum}
% \end{Key}\end{document}%%%%%%%%%%%%%%%%%%%%%%%%%
\alter Plants not dioecious; flowers bisexual, more than 2 per scape.
\alter Pedicels ebracteolate or only a few bracteolate at base.
\alter Pedicels equaling or slightly longer than perianth; ovary without concave nectaries at base; filaments 2/3--3/4 as long as perianth segments
\cname{灰皮薤}{grisellum}
\alter Pedicels more than 2 × as long as perianth; ovary with concave nectaries at base; filaments equaling or longer than perianth segments.
\alter Leaves flat; pedicels unequal; inner filaments with a rounded, irregularly denticulate lobe on each side near base
\cname{松潘薤}{songpanicum}
\alter Leaves semiterete; pedicels equal or subequal; inner filaments entire or with an entire tooth on each side near base.
\alter Ovary tuberculate; style slightly exserted
\cname{小山薤}{pallasii}
\alter Ovary smooth; style conspicuously exserted.
\alter Perianth pink to pale purple-red; filaments slightly longer than perianth segments
\cname{真籽薤}{eusperma}
\alter Perianth white; filaments ca. 2 × as long as perianth segments
\cname{茂汶薤}{maowenense}
\alter All pedicels subtended by bracteoles.
\alter Bulb tunic leathery, splitting along veins; perianth segments pale green with green midvein
\cname{沙地薤}{sabulosum}
\alter Bulb tunic membranous or papery, not splitting, or only apex splitting and becoming fibrous or reticulate; perianth segments white, pale red, red, or purple-red to dark purple, rarely pale green.
\alter Bulb ovoid or narrowly so.
\alter Leaves 3--5-angled in cross section, fistulose; scape lateral; base of inner filaments 1-toothed on each side
\cname{藠头}{chinense}
\alter Leaves semiterete or triangular in cross section, fistulose at least basally; base of inner filaments entire or occasionally 1-toothed on each side.
\alter Leaves 1--2 mm wide, semiterete; perianth white or pale red, sometimes greenish.
\alter Filaments slightly longer than to 1.5 × as long as perianth segments; ovary with concave nectaries at base covered by hoodlike projections
\cname{白花薤}{yanchiense}
\alter Filaments shorter or slightly longer than perianth segments; ovary without concave nectaries at base
\cname{尤尔都斯薤}{juldusicola}
\alter Leaves 2--5 mm wide, 3-angled or -keeled to obscurely 3-angled; perianth red to purple or lilac-pink to red-violet.
\alter Bulb tunic thinly leathery, split and becoming fibrous and subreticulate, rarely subentire; leaves keeled to obscurely 3-angled, rarely subfistulose near base; umbel globose, densely many flowered; perianth lilac-pink to red-violet
\cname{朝鲜薤}{sacculiferum}
\alter Bulb tunic membranous to scarious or subpapery, sometimes apex laciniate to fibrous; leaves 3-angled, subfistulose; umbel subfascicled to globose, laxly many flowered; perianth red to purple
\cname{球序薤}{thunbergii}
\alter Bulb ovoid-globose to subglobose (if ovoid, then ovary with concave nectaries at base without hoodlike projections).
\alter Pedicels equaling to slightly longer than perianth
\cname{头花薤}{glomeratum}
\alter Pedicels more than 2 × as long as perianth.
\alter Leaves 0.5--1.5 mm wide, terete; scape covered with leaf sheaths for 1/3--1/2 its length; ovary tuberculate
\cname{迷人薤}{delicatulum}
\alter Leaves 1--5 mm wide, flat, semiterete, or triangular-semiterete; scape covered with leaf sheaths for less than 1/3 its length; ovary smooth.
\alter Leaves flat; umbel without bulblets; filaments 1.5--2 × as long as perianth segments
\cname{唐古薤}{tanguticum}
\alter Leaves semiterete or triangular-semiterete; umbel $\pm$ with bulblets; filaments shorter than perianth segments
\cname{薤白}{macrostemon}
% \end{Key}\end{document}%%%%%%%%%%%%%%%
\alter Bulbs usually several in a cluster, cylindric, conical, or ovoid-cylindric, rarely ovoid; rhizomes well developed.
\alter Bulb tunic reticulate, subreticulate, or laxly fibrous.
\alter Perianth pale blue to blue or purplish blue.
\alter Filaments shorter than perianth segments.
\alter Perianth segments narrowly oblong to narrowly ovate-oblong, 11--14(--17) mm, margin entire; filaments usually ca. 4/5 as long as perianth segments; style usually 2--3 × as long as ovary
\cname{蓝花韭}{beesianum}
\alter Perianth segments ovate or ovate-oblong, 6--10 mm, at least margin of inner ones irregularly denticulate; filaments usually 1/2--2/3 as long as perianth segments; style shorter than or subequaling ovary.
\alter Perianth segments acuminate at apex, equal, irregularly denticulate at margin, rarely outer ones entire; leaves abaxially keeled, usually twisted when dry
\cname{齿被韭}{yuanum}
\alter Perianth segments obtuse at apex, inner ones longer and wider than outer, only inner ones irregularly denticulate at margin; leaves flat
\cname{高山韭}{sikkimense}
\alter Filaments longer than perianth segments.
\alter Leaves semiterete
\cname{天蓝韭}{cyaneum}
\alter Leaves flat.
\alter Pedicels extremely unequal, 2--4 × as long as perianth
\cname{异梗韭}{heteronema}
\alter Pedicels subequal, 1--2 × as long as perianth.
\alter Bulb tunic reticulate; umbel laxly few flowered; pedicels 1.5--2 × as long as perianth; base of inner filaments with 1 short tooth on each side, apex of tooth entire
\cname{疏花韭}{henryi}
\alter Bulb tunic subreticulate; umbel densely many flowered; pedicels 1--1.5 × as long as perianth; base of inner filaments with 1 long tooth on each side, apex of tooth sometimes denticulate
\cname{雾灵韭}{stenodon}
\alter Perianth white, pale red, purple-red, purple, dark purple, or yellow.
\alter Filaments more than 1.3 × as long as perianth segments.
\alter Perianth segments basally united for 1.5--2 mm into a short tube; filaments basally connate for 1.5--2 mm and adnate to perianth segments
\cname{管花韭}{siphonanthum}
\alter Perianth segments free; filaments connate only at base.
\alter Bulb tunic usually red, distinctly reticulate; inner filaments broadened for 1/3--1/2 their length; ovary without concave nectaries at base
\cname{青甘韭}{przewalskianum}
\alter Bulb tunic never red, reticulate or fibrous; inner filaments broadened for ca. 1/3 their length; ovary with concave nectaries at base.
\alter Perianth pale red or purple-red to purple.
\alter Bulb tunic fibrous, sometimes subreticulate; pedicels ebracteolate
\cname{多叶韭}{plurifoliatum}
\alter Bulb tunic reticulate; pedicels bracteolate at base.
\alter Leaves semiterete
\cname{细叶北韭}{clathratum}
\alter Leaves flat.
\alter Inner filaments entire
\cname{单丝辉韭}{schrenkii}
\alter Inner filaments with 1 or 2 teeth on each side.
\alter Perianth segments with red, slender midvein or without midvein; stigma punctiform
\cname{北韭}{lineare}
\alter Perianth segments with purple midvein; stigma capitate or subglobose.
\alter Perianth pale lilac to pinkish lilac
\cname{丽韭}{splendens}
\alter Perianth pink to pink-red
\cname{马克韭}{maackii}
\alter Perianth white to pale yellow.
\alter Leaves 2--7 mm wide, flat; teeth of inner filaments entire
\cname{新疆韭}{flavidum}
\alter Leaves 1--5 mm wide, semiterete, fistulose; teeth of inner filaments sometimes irregularly 2--4-denticulate.
\alter Leaves equaling to distinctly longer than scape; pedicels ebracteolate
\cname{阿拉善韭}{flavovirens}
\alter Leaves shorter than scape; pedicels bracteolate at base
\cname{白头韭}{leucocephalum}
\alter Filaments less than 1.3 × as long as perianth segments.
\alter Perianth yellow, later becoming red; filaments connate into tube for 3/5--4/5 their length
\cname{管丝葱}{semenovii}
\alter Perianth not yellow; filaments connate only basally or for 1/6--1/2 their length.
\alter Inner filaments toothed at base.
\alter Filaments ca. 1/2 as long as perianth segments
\cname{梭沙韭}{forrestii}
\alter Filaments slightly shorter than or equaling perianth segments.
\alter Leaves 3--5 mm wide, flat.
\alter Filaments slightly longer than perianth segments; stigma punctiform
\cname{直立韭}{amphibolum}
\alter Filaments slightly shorter than or equaling perianth segments
\cname{辉韭}{strictum}
\alter Leaves 0.25--1 mm wide, semiterete.
\alter Bulb tunic subreticulate; filaments connate into a tube for 1/6--1/2 their length, tube adnate to perianth segments for 1/3--1/2 its length.
\alter Perianth segments 6--8.5 mm; filaments connate for 1/3--1/2 their length
\cname{紫花韭}{subangulatum}
\alter Perianth segments 3--6 mm; filaments connate for 1/6--1/3 their length
\cname{碱韭}{polyrhizum}
\alter Bulb tunic distinctly reticulate; filaments connate only at base.
\alter Pedicels bracteolate at base; perianth pale purple to purple; ovary without concave nectaries at base
\cname{贺兰韭}{eduardii}
\alter Pedicels ebracteolate; perianth pale red; ovary with concave nectaries at base covered by hoodlike projections
\cname{针叶韭}{aciphyllum}
\alter Inner filaments entire at base.
\alter Scape covered with leaf sheaths for 1/4--1/2 its length.
\alter Pedicels subequal; leaves 2--5 mm wide
\cname{辉韭}{strictum}
\alter Pedicels unequal; leaves 0.5--1 mm wide.
\alter Bulb tunic distinctly reticulate
\cname{荒漠韭}{tekesicola}
\alter Bulb tunic subreticulate.
\alter Bulb tunic brown; inner filaments basally ca. 2 × as wide as outer ones, distally abruptly subulate; ovary with concave nectaries at base covered by hoodlike projections
\cname{褐皮韭}{korolkowii}
\alter Bulb tunic yellowish brown; inner filaments basally ca. 3 × as wide as outer ones, distally gradually attenuate; ovary with small, concave nectaries at base not covered by hoodlike projections
\cname{西疆韭}{teretifolium}
\alter Scape covered with leaf sheaths only at base.
\alter Bulb tunic reticulate or subreticulate; perianth white to pale red.
\alter Inner filaments broadly triangular, basally ca. 2 × as wide as outer ones; perianth segments with dark purple midvein
\cname{滩地韭}{oreoprasum}
\alter Inner filaments narrowly triangular, basally only slightly wider than outer ones; perianth segments without dark purple midvein.
\alter Scape 5--15 cm; stigma slightly 3-cleft
\cname{雪韭}{humile}
\alter Scape 25--60 cm; stigma punctiform.
\alter Leaves flat, solid; perianth segments white, usually with green midvein
\cname{韭}{tuberosum}
\alter Leaves triangular, abaxially keeled, fistulose; perianth segments white, rarely pale red, usually with pale red midvein
\cname{野韭}{ramosum}
\alter Bulb tunic fibrous or subreticulate at base; perianth dark purple, purple-red, or pale red.
\alter Inner filaments not broadened at base.
\alter Leaves 8--10 mm wide
\cname{宽叶滇韭}{rhynchogynum}
\alter Leaves 1--3 mm wide.
\alter Filaments 1/2--2/3 as long as perianth segments; ovary constricted at apex, without concave nectaries at base
\cname{滇韭}{mairei}
\alter Filaments slightly shorter or longer than perianth segments; ovary not constricted at apex, with concave nectaries at base covered by narrow, hoodlike projections
\cname{昌都韭}{changduense}
\alter Inner filaments broadened at base.
\alter Leaves 1.5--3(--5) mm wide, flat; perianth purple to dark purple
\cname{梭沙韭}{forrestii}
\alter Leaves 0.3--1.5 mm wide, semiterete to terete; perianth pale red to purple-red.
\alter Leaves distinctly longer than scape; umbel laxly few flowered; perianth segments 3--3.5 mm; ovary with concave nectaries at base
\cname{鄂尔多斯韭}{alabasicum}
\alter Leaves shorter than scape; umbel densely many flowered; perianth segments 6--9 mm; ovary without concave nectaries at base
\cname{蒙古韭}{mongolicum}
\alter Bulb tunic entire, never reticulate or only apical part becoming fibrous.
\alter Filaments shorter than perianth segments, never more than 4/5 as long as perianth segments.
\alter Perianth blue.
\alter Perianth segments narrowly oblong to narrowly ovate-oblong, 11--14(--17) mm, margin entire; filaments usually ca. 4/5 as long as perianth segments; style usually 2--3 × as long as ovary
\cname{蓝花韭}{beesianum}
\alter Perianth segments ovate or ovate-oblong, 6--10 mm, margin of at least inner ones irregularly denticulate; filaments usually 1/2--2/3 as long as perianth segments; style usually shorter than or subequaling ovary.
\alter Perianth segments acuminate at apex, equal, irregularly denticulate at margin, rarely outer ones entire; leaves abaxially keeled, usually twisted when dry
\cname{齿被韭}{yuanum}
\alter Perianth segments obtuse at apex, inner ones longer and wider than outer, only inner ones irregularly denticulate at margin; leaves flat
\cname{高山韭}{sikkimense}
\alter Perianth white, pale red, purple-red, pale purple, purple, yellow, or bright yellow.
\alter Perianth pale or bright yellow.
\alter Bulb cylindric; leaves slightly falcate, usually ca. 1/2 as long as or rarely subequaling scape, flat
\cname{折被韭}{chrysocephalum}
\alter Bulb ovoid-globose to ovoid; leaves subequaling scape, semiterete
\cname{金头韭}{herderianum}
\alter Perianth white, pale red, purple-red, pale purple, or purple.
\alter Filaments connate into an urceolate tube for 3/4--4/5 their length
\cname{坛丝韭}{weschniakowii}
\alter Filaments connate only at base or for 1/3--1/2 their length.
\alter Bulb tunic leathery; pedicels bracteolate at base; ovary with concave nectaries at base covered by hoodlike projections.
\alter Bulb tunic brown, apex becoming fibrous and somewhat subreticulate; pedicels unequal, 2--3 × as long as perianth in fruit
\cname{褐皮韭}{korolkowii}
\alter Bulb tunic light yellowish brown, apex laciniate; pedicels subequal, 1--2 × as long as perianth in fruit
\cname{丝叶韭}{setifolium}
\alter Bulb tunic membranous, papery, or thinly leathery; pedicels ebracteolate; ovary usually without concave nectaries at base (but present in \textit{A. pevtzovii}).
\alter Perianth segments 2.8--5 mm, apex of inner ones truncate or rounded-truncate.
\alter Pedicels unequal, 1.5--3.5 cm; perianth segments 3.9--5 mm
\cname{矮韭}{anisopodium}
\alter Pedicels subequal, 0.5--1.5 cm; perianth segments 2.8--4.2 mm.
\alter Bulb tunic laciniate; umbel hemispheric to globose, densely many flowered
\cname{雅韭}{elegantulum}
\alter Bulb tunic splitting at apex; umbel hemispheric to fascicled, laxly flowered
\cname{细叶韭}{tenuissimum}
\alter Perianth segments 6--18 mm, apex of inner ones obtuse or acuminate.
\alter Inner filaments ovate at base, 1-toothed on each side; ovary with concave nectaries at base
\cname{昆仑韭}{pevtzovii}
\alter Inner filaments subulate to narrowly ovate at base, entire; ovary without concave nectaries at base.
\alter Leaves 1.5--4 mm wide, flat; perianth segments 13--18 mm; filaments ca. 1/2 as long as perianth segments, all subulate
\cname{钟花韭}{kingdonii}
\alter Leaves 0.5--1 mm wide, semiterete; perianth segments 6--9 mm; filaments 2/3--3/4 as long as perianth segments, inner ones wider than outer.
\alter Bulbs densely clustered; bulb tunic papery, apex becoming fibrous; pedicels ca. 2 × as long as perianth; perianth segments ovate-oblong, 7--9 × 2.5--3 mm; broadened base of inner filaments narrowly triangular
\cname{永登韭}{yongdengense}
\alter Bulbs laxly clustered; bulb tunic membranous, laciniate; pedicels 1--1.5 × as long as perianth; perianth segments elliptic to ovate, 6--7 × 3--4 mm; broadened base of inner filaments narrowly ovate
\cname{疏生韭}{caespitosum}
\alter Filaments slightly shorter to longer than perianth segments.
\alter Leaves flat or keeled abaxially.
\alter Bulbs attached to a stout, horizontal or oblique rhizome; scape usually 2-angled.
\alter Leaves abaxially keeled; ovary with concave nectaries at base covered by hoodlike projections
\cname{泰山韭}{taishanense}
\alter Leaves flat; ovary without concave nectaries at base.
\alter Perianth white to yellow; pedicels ebracteolate
\cname{冀韭}{chiwui}
\alter Perianth pale red or pale purple to purple-red; pedicels bracteolate at base.
\alter Leaves straight or spirally tortuous, (1.5--)2--6 mm wide.
\alter Leaves straight, 1.5--4 mm wide
\cname{岩韭}{spurium}
\alter Leaves spirally tortuous, 4--6 mm wide
\cname{扭叶韭}{spirale}
\alter Leaves straight or slightly falcate, 6--15 mm wide.
\alter Inner filaments entire at base
\cname{山韭}{senescens}
\alter Inner filaments 1-toothed on each side at base
\cname{齿丝山韭}{nutans}
\alter Bulbs attached to a vertical rhizome; scape terete.
\alter Perianth yellowish green to pale yellow.
\alter Scape covered with leaf sheaths for ca. 1/2 its length; pedicels 2--4 × as long as perianth, bracteolate at base
\cname{高葶韭}{obliquum}
\alter Scape covered with leaf sheaths at base only; pedicels 1--1.5 × as long as perianth, ebracteolate.
\alter Inner filaments entire at base; umbel globose, densely many flowered; leaves 3--8 mm wide
\cname{野黄韭}{rude}
\alter Inner filaments 1-toothed on each side at base; umbel laxly flowered; leaves 2--3 mm wide
\cname{矮齿韭}{brevidentatum}
\alter Perianth white, pale red, purple-red, purple, or blue.
\alter Inner filaments with 1 irregularly denticulate tooth on each side at base; ovary with concave nectaries at base covered by hoodlike projections.
\alter Leaves broadly linear to linear-lanceolate, 5--23 mm wide; spathe with a long beak sometimes to 7 cm; perianth white
\cname{天蒜}{paepalanthoides}
\alter Leaves linear, 2--6(--8) mm wide; spathe with a short beak; perianth pale red, pale purple, or purple
\cname{多叶韭}{plurifoliatum}
\alter Inner filaments entire at base or, if toothed, teeth never denticulate; ovary with concave nectaries at base not covered by hoodlike projections.
\alter Spathe with long beak
\cname{条叶长喙韭}{kurssanovii}
\alter Spathe with short beak.
\alter Bulb tunic blackish gray to black, papery or membranous.
\alter Leaves straight; perianth pink or pinkish lilac; ovary with concave nectaries at base
\cname{宽苞韭}{platyspathum}
\alter Leaves falcate; perianth white; ovary without concave nectaries at base
\cname{帕里韭}{phariense}
\alter Bulb tunic red-brown or brown to yellowish brown, leathery or thinly so.
\alter Leaves 1--1.5(--3) mm wide, margin scabrous-denticulate; ovary without concave nectaries at base
\cname{草地韭}{kaschianum}
\alter Leaves 2--17 mm wide, margin smooth; ovary with concave nectaries at base.
\alter Bulb tunic red-brown, lustrous
\cname{北疆韭}{hymenorhizum}
\alter Bulb tunic brown to yellowish brown, dull.
\alter Leaves usually falcate; inner filaments broadened at base
\cname{镰叶韭}{carolinianum}
\alter Leaves usually not falcate; inner filaments not broadened at base
\cname{白韭}{blandum}
\alter Leaves semiterete or terete, solid or fistulose, adaxially channeled.
\alter Inner filaments obtusely 1-toothed on each side at base.
\alter Perianth white or pale yellow; style conspicuously exserted; ovary with concave nectaries at base
\cname{阿拉善韭}{flavovirens}
\alter Perianth red, pale purple-red, or purple-red; style not exserted; ovary without concave nectaries at base.
\alter Inner filaments broadened for ca. 4/5 their length
\cname{砂韭}{bidentatum}
\alter Inner filaments broadened for ca. 1/2 their length
\cname{短齿韭}{dentigerum}
\alter Inner filaments entire at base.
\alter Beak of spathe several times as long as limb.
\alter Perianth segments pale yellow with green midvein
\cname{石坡韭}{petraeum}
\alter Perianth segments purple-red to pale red, rarely white, with dark midvein
\cname{长喙韭}{saxatile}
\alter Beak of spathe shorter than or equaling limb.
\alter Perianth white, pale yellow, or greenish yellow.
\alter Leaves solid, margin ciliate-denticulate
\cname{天山韭}{tianschanicum}
\alter Leaves fistulose, margin smooth.
\alter Bulb tunic red-brown, lustrous; scape solid; pedicels bracteolate at base
\cname{黄花韭}{condensatum}
\alter Bulb tunic light brown to brown, dull; scape fistulose; pedicels ebracteolate
\cname{西川韭}{xichuanense}
\alter Perianth pale red, red, pale purple, or purple.
\alter Pedicels bracteolate at base.
\alter Ovary with longitudinally convex nectaries along septa; nectary pit at ovary base open
\cname{蜜囊韭}{subtilissimum}
\alter Ovary without nectaries or only with concave nectaries at base.
\alter Bulbs attached to a robust, horizontal rhizome; filaments equaling or slightly longer than perianth segments; ovary without concave nectaries
\cname{蒙古野韭}{prostratum}
\alter Bulbs attached to vertical rhizome; filaments 1.25--2 × as long as perianth segments; ovary with concave nectaries.
\alter Bulb tunic red-brown, lustrous, scarious to subleathery; leaves 2--3 mm wide, margin smooth
\cname{长柱韭}{longistylum}
\alter Bulb tunic brown, dull, leathery; leaves 0.5--2 mm wide, margin ciliate- or scabrous-denticulate.
\alter Bulb tunic leathery, entire or laciniate at apex; ovary with concave nectaries covered by hoodlike projections
\cname{石生韭}{caricoides}
\alter Bulb tunic thinly leathery to papery, splitting; ovary with concave nectaries not covered by hoodlike projections.
\alter Bulbs to 10 cm; tunic yellowish brown; filaments ca. 1/2 as long as perianth segments
\cname{大鳞韭}{megalobulbon}
\alter Bulbs shorter; tunic brown; filaments 1.25--1.5 × as long as perianth segments
\cname{天山韭}{tianschanicum}
\alter Pedicels ebracteolate.
\alter Inner filaments conspicuously broadly ovate or ovate-oblong at base.
\alter Bulb tunic grayish white, sometimes reddish; pedicels 2--3 × as long as perianth; perianth purple-red; style slightly longer than ovary
\cname{短齿韭}{dentigerum}
\alter Bulb tunic brown; pedicels shorter than or equaling perianth; perianth pale red; style shorter than ovary
\cname{砂韭}{bidentatum}
\alter Inner filaments subulate to narrowly triangular at base, gradually attenuate toward apex.
\alter Scape covered with leaf sheaths for 1/3--1/2 its length; leaves 1 or 2, margin smooth; perianth pink
\cname{少花葱}{oliganthum}
\alter Scape covered with leaf sheaths only at base; leaves more than 3, margin scabrid; perianth pale purple or red-purple.
\alter Inner perianth segments irregularly denticulate at distal margin and apex; style exserted
\cname{蒙古野韭}{prostratum}
\alter Inner perianth segments entire; style not exserted
\cname{红花韭}{rubens}
\end{Key*}

\end{document}
